\begin{document}

\title{Piano di qualifica}


\section{Introduzione}

\subsection{Scopo del documento}
Il presente documento definisce il livello di qualit\`a atteso del prodotto e del processo di realizzazione dello stesso, indica le metodologie da adottare per assicurare la qualit\`a e riporta le attivit\`a svolte inerenti la gestione della qualit\`a.

\subsection{Scopo del prodotto}
Il prodotto intende permettere la realizzazione di \underline{grafica vettoriale} tramite una comoda interfaccia web.

\subsection{Glossario}
Si veda \ref{Glossario.pdf}

\subsection{Riferimenti}


\section{Visione generale della strategia di verifica}

\subsection{Organizzazione, pianificazione strategica e temporale, responsabilit\`a}
Le attivit\`a di verifica si dovranno svolgere immediatamente in seguito ad una modifica, o ad un insieme di modifiche. Qualora sia possibile svolgere l'attivit\`a in modo automatico, essa dovr\`a essere applicata in seguito ad ogni singolo cambiamento. I test automatici saranno predisposti dai verificatori. Negli altri casi, onde evitare di attivare un'intero ciclo di verifica con gli annessi costi per delle modifiche di scarso rilievo, l'attivit\`a sar\`a svolta quando le differenze tra la versione attuale del prodotto e la versione su cui \'e stata applicata l'ultima verifica sono sufficientemente rilevanti. \'E compito di chi effettua le modifiche di segnalare ai verificatori la necessit\`a di un'attivit\`a di verifica. Il responsabile della qualifica dovr\`a accertarsi che il tutto si svolga correttamente e che segua quanto descritto nel presente documento.
\end{document}